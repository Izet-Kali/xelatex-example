\section{Рисунки}

\begin{figure}[H]
 	\centering
 	    \fbox{\includegraphics[width=0.7\linewidth]{test.png}}
 	    \caption{test}
 	    \label{image:test}
\end{figure}


\section{Блок-схемы}
%block-diagram
\begin{figure}[H]
\centering
\fbox{\begin{tikzpicture}[node distance = 2.5cm, auto]

\node[main] (o1) {Установка полетного контроллера}

\node[main, below of = o1] (o2) {Установка контроллеров двигателя}
\node[main, below of = o2] (o3) {Установка приемного модуля}
\node[main, below of = o3] (o4) {Монтаж кабелей}



\draw[->,very thick] (o1) -- (o2);
\draw[->,very thick] (o2) -- (o3);
\draw[->,very thick] (o3) -- (o4);

\end{tikzpicture}}
\caption{ }
\label{image:test1}
\end{figure}


%diagram
\begin{tikzpicture}
    \graph[nodes={align=center}, grow down sep, branch right sep]{
    1 -> 
    {
    2, 3
    }
    };
\end{tikzpicture}



% Принципиальные и не очень схемы
\begin{figure}[H]
\centering
\fbox{\begin{circuitikz}[node distance = 2cm]
%\ctikzset{arrow={Triangle}}
\ctikzset{european}


 
%\node[]{}(t1)  \node[left of = t1]{$Q_{(t)}$};
\node[bareRXantenna, box only, fill=cyan, anchor=waves](Antena){2.4 - 2.6 ГГц}
%\node[waves, left of = Antena](waves){2.4 - 2.6 ГГц}

\node[below of = Antena, twoportshape, t =ПМ, boxed, fill = yellow!95!black,] (n0){}

\node[twoportshape, below of = n0, t = ПК, fill= yellow!95!black,] (n1){}
\node[twoportshape, right of = n1, xshift = 2 cm, t = ESC, fill= yellow!10!green] (esc1) {}
\node[twoportshape, below of = esc1, t = ESC, fill= yellow!10!green] (esc2) {}
\node[twoportshape, below of = esc2, t = ESC, fill= yellow!10!green] (esc3) {}
\node[twoportshape, below of = esc3, t = ESC, fill= yellow!10!green] (esc4) {}


\node[elmech, right of= esc1, fill=gray!15](m1){M1}
\node[elmech, right of= esc2, fill=gray!15](m2){M2}
\node[elmech, right of= esc3, fill=gray!15](m3){M3}
\node[elmech, right of= esc4, fill=gray!15](m4){M4}


\draw[->,Triangle] (Antena) --  (n0);
\draw[->,Triangle] (n0) -- (n1);
\draw[->,Triangle] (n1) -- (esc1); \draw[->,Triangle] (esc1) -- (m1);
\draw[->,Triangle] (n1) |- (esc2); \draw[->,Triangle] (esc2) -- (m2);
\draw[->,Triangle] (n1) |- (esc3); \draw[->,Triangle] (esc3) -- (m3);
\draw[->,Triangle] (n1) |- (esc4); \draw[->,Triangle] (esc4) -- (m4);

\end{circuitikz}

}
\caption{Структурная схема}
\label{image:mr1}
\end{figure}




% Таблицы


\begin{longtable}{|p{6cm}|p{3cm}|p{5cm}|}
\hline
\textbf{Название} & \textbf{Сокращение} & \textbf{Децимальный номер}\\
\hline
\endfirsthead    % Первые ячейки начала таблицы
\hline
\textbf{Название} & \textbf{Сокращение} & \textbf{Децимальный номер}\\
\hline
\endhead

АКБ & & \\ \hline
Рама - SPARKHOBBY F450 & &\\ \hline
Мотор - Emax ECOII 2306 2400kv & &\\ \hline
ESC - контроллер двигателя & КД, ESC & 466369\\ \hline
Полетный контроллер & ПК & \\ \hline
Приемник - ELRS mini 2.4 ГГц RX SX1280 & ПМ& \\ \hline

\end{longtable}


% формулы

\begin{equation}
    X_c = \frac{1}{j\omega C}
\end{equation}

\begin{equation*}
    X_c = \frac{1}{j\omega C}
\end{equation*}



% графики

% график по координатам



\begin{figure}[H]
    \centering
    \fbox{
        \begin{tikzpicture}
            \begin{axis}[grid = major,
            %grid = both,
                xlabel = {$f$ [ГЦ]},
                ylabel = {$U$ [Вольт]},
                %xmin = 0,
                %xmax = 30,
            ]
            \addplot[black] coordinates {
(50,1.00039259179737)
(100,1.00157609427361)
(150,1.00356790340693)
(200,1.00639774350291)
(250,1.0101088188689)
(300,1.01475954313736)
(9700,0.643857114937704)
(9750,0.643963363580767)
(9800,0.644067903632787)
(9850,0.644170772013437)
(9900,0.644272004637651)
(9950,0.64437163644866)
(10000,0.644469701449786)
};
        
            \end{axis}
        \end{tikzpicture}}
\end{figure}


\lstset{language=c++, extendedchars=true, numbers=left, frame=single, frameround = tttt, keepspace=true, breaklines=true}

\section{Листинг кода}


\subsection{Из файла} 
\lstset{language=C++,numbers=left, stepnumber=1, numberstyle=\normalfont, numbersep=15pt, firstnumber = 1, linerange=1-100, frame=single, breaklines=true, framesep=10pt, xleftmargin=1em}
\lstinputlisting[caption={Пример большого листинга},label={lstX}]{Gradient_file.m}

\lstinputlisting[caption = test, label ={code:test1}]{code/3.cpp}

\subsection{Самописный}

\begin{lstlisting}[caption = test]
    #include <iostream>

    using namespace std;
\end{lstlisting}


\subsection{Разделение текста на колонки}
\begin{minipage}{.49\textwidth}
Раз колонка. И какой-то текст.
\end{minipage}
\begin{minipage}
 два колонка.
\end{minipage}





\section{Красочные блоки}



\begin{tcolorbox}[%watermark graphics=Artificer.png,
                  %watermark opacity=0.25,
                  enhanced jigsaw,
                  %skin=bicolor,
                  widget,
                  colback=black!2!white,% background
                  %colbacklower=yellow!20!white,
                  width=.49\linewidth,% Use 5cm total width,
                  arc=1mm, auto outer arc,
                  boxrule=4pt,
                  frame style={left color=red!75!black, right color=blue!75!black}, %наложение градиента
                  colframe=white!75!black,% black frame colour
                  %frame style image=3.jpg,
                  fonttitle=\bfseries\centering,
                  colbacktitle=black!75!black!1!white,
%                  drop shadow={Maroon!50!gray!80}
                %height = 60mm,
                title=Что-то,
                coltitle=black,attach boxed title to top center=
{yshift=-0.25mm-\tcboxedtitleheight/2,yshifttext=2mm-\tcboxedtitleheight/2},
boxed title style={boxrule=0.5mm,
frame code={ \path[tcb fill frame] ([xshift=-4mm]frame.west)
-- (frame.north west) -- (frame.north east) -- ([xshift=4mm]frame.east)
-- (frame.south east) -- (frame.south west) -- cycle; },
interior code={ \path[tcb fill interior] ([xshift=-2mm]interior.west)
-- (interior.north west) -- (interior.north east)
-- ([xshift=2mm]interior.east) -- (interior.south east) -- (interior.south west)
-- cycle;} }]


лалала

\tcblower
лалала

\end{tcolorbox}



Все это дело можно вынести в преамбулу командой по типу:

\newtcolorbox{mybox}[1]{
	title={#1},
	%watermark graphics=Artificer.png,
                  %watermark opacity=0.25,
                  enhanced jigsaw,
                  %skin=bicolor,
                  widget,
                  colback=black!2!white,% background
                  %colbacklower=yellow!20!white,
                  width=.49\linewidth,% Use 5cm total width,
                  %arc=1mm, auto outer arc,
                  boxrule=2pt,
		  sharp corners,
                  frame style={left color=red!75!black, right color=blue!75!black}, %наложение градиента
                  colframe=white!75!black,% black frame colour
                  %frame style image=3.jpg,
                  fonttitle=\bfseries\centering,
                  colbacktitle=black!75!black!1!white,
%                  drop shadow={Maroon!50!gray!80}
                %height = 60mm,
}



Или как я понимаю задать как настройки командой %\tcbset{}

\begin{mybox}{title}
 Соответственно туть можно что-то написать хехе))


 \tcblower


 И тут как бы тоже))
\end{mybox}




С этим пакетом можно делать разные приколы, например environment upper=itemize позволяет внутри фрейма делать список:






\begin{tcolorbox}[
	title={#1},
	%watermark graphics=Artificer.png,
                  %watermark opacity=0.25,
                  enhanced jigsaw,
                  %skin=bicolor,
                  widget,
                  colback=black!2!white,% background
                  %colbacklower=yellow!20!white,
                  width=.49\linewidth,% Use 5cm total width,
                  %arc=1mm, auto outer arc,
                  boxrule=2pt,
		  sharp corners,
                  frame style={left color=red!75!black, right color=blue!75!black}, %наложение градиента
                  colframe=white!75!black,% black frame colour
                  %frame style image=3.jpg,
                  fonttitle=\bfseries\centering,
                  colbacktitle=black!75!black!1!white,
%                  drop shadow={Maroon!50!gray!80}
                %height = 60mm,

environment upper=itemize]

	\item 12345687
	\item qwe

\end{tcolorbox}


ИИИ это не работает почему-то




Также можно делать таблички


% это в преамбулу точно))
\newcolumntype{Y}{>{\raggedleft\arraybackslash}X}% see tabularx



\begin{tcolorbox}[enhanced,fonttitle=\bfseries\large,fontupper=\normalsize\sffamily,
colback=yellow!10!white,colframe=red!50!black,colbacktitle=Salmon!30!white,
coltitle=black,center title, tabularx={X||Y|Y|Y|Y||Y},title=My table]
Group & One & Two & Three & Four & Sum\\\hline\hline
Red & 1000.00 & 2000.00 & 3000.00 & 4000.00 & 10000.00\\\hline
Green & 2000.00 & 3000.00 & 4000.00 & 5000.00 & 14000.00\\\hline
Blue & 3000.00 & 4000.00 & 5000.00 & 6000.00 & 18000.00\\\hline\hline
Sum & 6000.00 & 9000.00 & 12000.00 & 15000.00 & 42000.00
\end{tcolorbox}

\section{Карточки}

Если идти через этот прикол, то мне понравилось в свое время через них делать карточки, тут можно вставлять картинки и как ватермарки и на фон например обрамления.

\dcartspel{\small Погребальный звон}{
\small Вы указываете на одно существо, которое можете видеть в пределах дистанции. Цель должна преуспеть в спасброске Мудрости, иначе получит 1к8 урона некротической энергией. Если хиты цели были не полные, то вместо 1к8 она получает 1к12 урона некротической энергией.

}{12345687}
%%%%




