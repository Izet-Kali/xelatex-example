\usepackage{extsizes}
\usepackage{cmap} % для кодировки шрифтов в pdf
\usepackage[T2A]{fontenc}
\usepackage[utf8]{inputenc}
\usepackage[russian]{babel}%Языки используемые в документе
\usepackage{ucs}
\usepackage[pdftex]{graphicx} % для вставки картинок
\graphicspath{{images/}}%Папка с изображениями
\DeclareGraphicsExtensions{.pdf,.png,.jpg,.bmp}%Установка форматов изображений


\usepackage[nooneline]{caption}% Переопределение заголовков рисунков и таблиц
%\RequirePackage{caption}
	\DeclareCaptionLabelSeparator{defffis}{ -- }%Установка дефиса
	

\captionsetup[table]{labelsep=defffis,justification=raggedright,singlelinecheck=false,font=normalsize}
\captionsetup[figure]{justification=centering,labelsep=defffis,name=Рисунок} 






\usepackage{floatrow} % плавающие обьекты, крутые оформления названий рисунков и тп.
\floatsetup[table]{style=Plaintop} %Название таблицы сверху
\usepackage{setspace} 
\usepackage{nomencl} % список сокращений
\usepackage{pdfpages} %для вставки пдф страниц
\usepackage{longtable} % Подключение переносимых автоматически таблиц
\usepackage{amssymb,amsfonts,amsmath,amsthm} % математические дополнения от АМС
\usepackage{indentfirst} % отделять первую строку раздела абзацным отступом тоже
\usepackage{color} % названия цветов [usenames,dvipsnames,table,xcdraw,hyperref]
%\usepackage[table,xcdraw,hyperref]{xcolor} %цвета
%\usepackage{makecell}
%\usepackage{multirow} % улучшенное форматирование таблиц
\usepackage{ulem} % подчеркивания
\usepackage{import}% Импорт файлов
\usepackage{float} % Расширенное управление плавающими объектами
\usepackage{lscape}  % Повороты листа

%Графики и схемы
\usepackage{tikz} % Рисование графиков и тд.
\usetikzlibrary{graphs}
% Блок-схема определяет основную форму
\tikzstyle{main} = [rectangle, minimum width = 2cm, minimum height=1cm,text centered, draw = black]
\tikzstyle{mainr} = [rectangle, rounded corners, minimum width = 2cm, minimum height=1cm,text centered, draw = black]
 % Форма стрелки
\tikzstyle{arrow} = [ultra thick,->,>=stealth]

% Графики функций
\usepackage{pgfplots}
\pgfplotsset{compat=1.9}



% Определение размеров полей
\usepackage{geometry}
	\geometry{left=3cm}
	\geometry{right=1.5cm}
	\geometry{top=2cm}
	\geometry{bottom=2cm}
		
%Установка шрифтов
\usepackage{xltxtra} % fontspec %пакет хелатех для таймс нью роман
\setmainfont{Times New Roman} % таймс нью роман
%\defaultfontfeatures{Ligatures=TeX,Mapping=tex-text}

\frenchspacing
\pagestyle{plain}
\linespread{1} % полуторный интервал
\setlength{\parindent}{1.27cm} % Абзацный отступ
\setlength{\parskip}{0pt}
\sloppy % указывает, что с залезанием слов на поля следует бороться, даже применяя недопустимо длинные пробелы.

\usepackage{hyperref} %Ссылки
\definecolor{linkcolor}{HTML}{000000} % цвет ссылок (черный)
\definecolor{urlcolor}{HTML}{0000ff} % цвет гиперссылок

 
%Библиография
%\usepackage{biblatex}
\usepackage[square,numbers,sort&compress]{natbib}
\renewcommand{\bibnumfmt}[1]{#1.\hfill} % нумерация источников в самом списке — через точку
\renewcommand{\bibsection}{\centering{\section*{Список использованных источников}}} % заголовок специального раздела

% изменял 12.04.2024
\setlength{\bibsep}{1pt} %было 0pt и писало: 
%Undefined control sequence.
%Missing number, treated as zero
% Illegal unit of measure (pt inserted)

%\bibliographystyle{gost780u}
\bibliographystyle{unsrt} % стиль цитат
%\bibliographystyle{gost-numeric}
%\bibliographystyle{vak} % стиль под гост с https://academics.hse.ru/bibliography/bibtex_manual/2.4.bibtex_styles


\usepackage{tocloft} %регулировка расположения TableOfContent (Оглавления) на странице
% % Отточия в Оглавлении
%\renewcommand\cftchapdotsep{\cftdot} %добавляет отточия после \chapter{title}
\renewcommand\cftsecdotsep{\cftdot} %делает отточия после \section{title} частыми.



\makeatletter
\renewcommand{\l@section}{\@dottedtocline{1}{0em}{1.25em}}
\renewcommand{\l@subsection}{\@dottedtocline{2}{1.25em}{1.75em}}
\renewcommand{\l@subsubsection}{\@dottedtocline{3}{2.75em}{2.6em}}
\makeatother

\usepackage{titlesec}    

% красивый блок настроек

%Раньше писал \shape перед \thesection и т.д., не понимаю зачем это, выдавала ошибку - непонятное выражение
\titleformat{\section}[block]{\Large\bfseries\filcenter}{\thesection}{1em}{}
\titleformat{\subsection}[block]{\large\normalfont\bfseries\filcenter}{\thesubsection}{1em}{}
\titleformat{\subsubsection}[block]{\normalfont\bfseries\filcenter}{\thesubsubsection}{1em}{}
\titleformat{\paragraph}[block]{\normalfont\bfseries}{\theparagraph}{1em}{}
\titleformat{\subparagraph}[runin]{\normalfont\bfseries}{\thesubparagraph}{1em}{}

%%%

%%%
% Кастомный блок настроек для заметок
%\titleformat{name=\section}[block]
%  {\color{black}\bfseries}
%  {}
%  {0pt}
%  {\colorsection}
%\titleformat{name=\subsection}[block]
%  {\color{black}\bfseries}
%  {}
%  {0pt}
%  {\colorsubsection}
  
%\newcommand{\colorsection}[1]{%
%  \fcolorbox{black}{cyan!20}{\parbox{\dimexpr\columnwidth-2\fboxsep-2\fboxrule\relax}{\strut~#1\strut}}}
%\newcommand{\colorsubsection}[1]{%
%  \fcolorbox{black}{yellow!25}{\parbox{\dimexpr\columnwidth-2\fboxsep-2\fboxrule\relax}{\strut~#1\strut}}}
%%%%



%\S\thesection
% оформляет номер как парагрф Ы




%\titleformat{\subsubsection}[block]{\normalfont\bfseries\filcenter}{}{1em}{}


%Блок настроек, который с меня требовал щучкин в бакалавриате

%\titlespacing{\section}{0pt}{1.4pt}{1.4pt}
%\titlespacing{\subsection}{0pt}{1.4pt}{1.4pt}
%\titlespacing{\subsubsection}{0pt}{1.4pt}{1.4pt}


%\titleformat{\section}[block]{\filcenter}{\centering}{\thesection}{\textbf}
%\titleformat{\subsection}[block]{\filcenter}{\centering}{\thesubsection}{\textbf}
%\titleformat{\subsubsection}[block]{\filcenter}{\centering}{}{\textbf}

% Тоже самое с параграфами
%\titleformat{\paragraph}[block]{\normalfont\bfseries}{\theparagraph}{1em}{}
%\titleformat{\subparagraph}[runin]{\normalfont\bfseries}{\thesubparagraph}{1em}{}

%\renewcommand{\aftersection}{6pt plus .1pt}

%Списки

%Настройка отступов в списках enumerate
\makeatletter
\let\old@enumerate=\enumerate
\def\enumerate{\old@enumerate
\setlength{\itemsep}{0pt}
\setlength{\parskip}{0pt}
\setlength{\leftskip}{0pt}
}\makeatother


%Настройка отступов в списках itemize
%\makeatletter
%\let\old@itemize=\itemize
%\def\itemize{\old@itemize
%\setlength{\itemsep}{0pt}
%\setlength{\parskip}{0pt}
%\setlength{\leftskip}{0pt}
%\makeatother

%маркеры списков
%\renewcommand{\labelitemi}{--}      % Маркер списка --
%\renewcommand{\labelenumi}{--}      % Список второго уровня
%\renewcommand{\labelenumii}{--}     % Список третьего уровня
    %\renewcommand{\labelenumi}{\asbuk{enumi})}    % Список второго уровня
    %\renewcommand{\labelenumii}{\arabic{enumii})} % Список третьего уровня
%Нумерация цифрами
\renewcommand{\labelenumii}{\arabic{enumi}.\arabic{enumii}.}



\usepackage{circuitikz} %Принципиальные схемы

%Листинг кода
\usepackage{listings}

\lstset{
	language=C++,
	numbers=left, 
	stepnumber=1, 
	numberstyle=\normalfont,
	numbersep=15pt,
	firstnumber = 1,
	linerange=1-100,
	frame=single,
	frameround = tttt,
	breaklines=true,
	framesep=10pt,
	xleftmargin=1em,
	keepspace=true,
	%extendedchars=true,
	commentstyle=\scriptsize}


\usepackage{bookmark} %%
