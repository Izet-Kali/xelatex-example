Почта группы: fkmo-01-22@mail.ru


Пароль: magistratura0122

\section{Практика I}
\paragraph{Ознакомительная}
С 09.02.2023 по 14.06.2023

Состав отчета
\begin{enumerate}
    \item Проектная часть
    Проект работы, описание команды, взаимодействия внутри команды, межкультурные взаимодействия.
    \item Практическая часть
    Аналитический обзор, патентное исследование, обзор производителей, разработка тз.
\end{enumerate}

\section{Практика II}
\paragraph{Проектно-технологическая}
\begin{enumerate}
    \item Проектная часть
    Проект работы, описание команды, взаимодействия внутри команды, межкультурные взаимодействия.
    \item Практическая часть
    Разработка схемы; Моделирование; Расчеты; Описание конструкции; Разработка КД.
\end{enumerate}

\section{Практика III}

\paragraph{НИР}

\begin{enumerate}
    \item Проектная часть
    Проект работы, описание команды, взаимодействия внутри команды, межкультурные взаимодействия.
    \item Практическая часть
    Разработка технологии изготовления; Разработка технологии сборки устройства; Технологические расчеты; Разработка и оформление МК.
\end{enumerate}



\section{Практика IV}

\paragraph{Преддипломная}
По сути вся ВКР
\begin{enumerate}
    \item Проектная часть
    Проект работы, описание команды, взаимодействия внутри команды, межкультурные взаимодействия, внешние взаимосвязи.
    \item Во введении должны быть отраженны актуальность и новизна.
    \item Аналитический обзор.
    \item Конструкторско-технологическая часть.
    \item Экспериментальная часть.
    \item Заключение.
    Краткие выводы по результатам выполненной работы или отдельных этапов. Оценка полноты решений поставленных задач, разработка рекомендаций и исходных данных по конкретному использованию результатов работы. Результаты оценки технико-экономических показателей, результаты оценки уровня выполненной работы в сравнении с лучшими достижениями в этой области.
\end{enumerate}


\section{Правки в дисертацию}

\begin{enumerate}
    \item Название части \colorbox{red!50}{Постановочно-экспериментальная} заменить на \colorbox{green!50}{Экспериментальная}.
    \item Позиции на сборочное чертеже проставляются точной, а не стрелкой.
    \item Расчет технологичности перефразировать в формат того, что расчет сделан, чтобы оценить технологичность, вывод из этого что девайс можно производить мелкосерийно.
    \item Добавить легенду на графики.
    \item Климатические добавить главу климатические испытания.
    \item В таблице сравнения добавить ячейку соответствия требования.
    \item В конце испытаний сделать вывод, в выводе работыв упоминуть испытания и итог испытаний.
    \item Децимальные номера.
    \item титульник в презентацию.
    \item Титульник к заданию ВКР.
    \item В оглавлении убрать номера приложений.
    \item Команда проекта добавить функций инженеру-конструктору
    \item В патентный поиск добавить итоговую таблицу по госту.
    \item 1 можно увеличить шрифт.
    \item 1.1 токо жирный, шрифт не увеличивать.
    \item 1.1.1 даже дирным не сделать.
    \item знаки препинания в формулах.
    \item схема Э1 в расчетах.
    \item Сборочный чертеж печатной платы переделать. 
\end{enumerate}